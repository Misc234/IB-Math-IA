\documentclass[11pt]{article}

\begin{document}
Inline math notation is done this way: $(x+4)$ \\
Double back slash is a line-break....\\
\\
Display math mode is don with double dollar signs
$$y=a+bx$$
Superscripts: $2x^3$\\
In superscripts only the first character will be scripted unless they are put into
parentheses. $$2x^{34+t}$$
Subscripts use underscore: $x_2$ and $x_{12}$ and ${x_{12}}_2$\\
Greek letters: Use backslash and spell out the name, A-L-E-X-A-N-D-E-R We are meant to be $A=\pi r^2$\\
Trig is this: $\sin(x)$, $\cos(x)$, $\tan(x)$\\
Logs: $\log_a{b}$, and $\ln{x}$\\
Roots: $\sqrt{2}$ and $\sqrt[3]{5}$ and $\sqrt{1+\sqrt{x}}$
\\
Fractions: $\displaystyle{ \frac{x+4}{x^2}} $ and $$\frac{x+4}{x^2}$$
to display parentheses use back slashes: §\{ \}§\\
same goes for the dollar sign: $\$5.98$\\
fractions with brackets: $$3\left(\frac{3}{4+x}\right)$$
$$3\left[\frac{3}{4+x}\right]$$
$$3\left|\frac{3}{4+x}\right|$$
$$3\left\{\frac{3}{4+x}\right\}$$\\
Hide elements using '.':
$$3 \left\{ \frac{3}{4+x} \right.$$\\

Tables:\\ \\
\begin{tabular}{c|cc}
	$x$ 	& 1 	& 2 \\ \hline
	$f(x)$	& 10	& 15
\end{tabular}

Equation arrays: \\
\begin{eqnarray*}
	21x^2 + 3x + 2 &=& 3x + 4 \\
	21x^2 &=& -2 \\
	x^2 &=& \frac{-2}{21} \\
	x &=& \pm \sqrt{\frac{-2}{21}}
\end{eqnarray*}

\begin{enumerate}
	\item blah
	\item blah
	\item blah
\end{enumerate}
\begin{itemize}
	\item blah
	\item blah
	\item blah
\end{itemize}
\textbf{bold}
\textit{italic}
\textit
\end{document}

