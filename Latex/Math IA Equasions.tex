\documentclass[11pt]{article}
\usepackage{graphicx}
\usepackage{amsmath}
\usepackage{amsfonts}
\usepackage[margin=2cm]{geometry}
\usepackage{fancyhdr}
\usepackage{enumitem}
\usepackage{natbib}
\usepackage[utf8]{inputenc}
\usepackage{float}

\pagestyle{fancy}
\numberwithin{equation}{subsection}

\makeatletter % used for using '&' to align subscripts
\newcommand{\subalign}[1]{%
  \vcenter{%
    \Let@ \restore@math@cr \default@tag
    \baselineskip\fontdimen10 \scriptfont\tw@
    \advance\baselineskip\fontdimen12 \scriptfont\tw@
    \lineskip\thr@@\fontdimen8 \scriptfont\thr@@
    \lineskiplimit\lineskip
    \ialign{\hfil$\m@th\scriptstyle##$&$\m@th\scriptstyle{}##$\crcr
      #1\crcr
    }%
  }
}
\makeatother

%DECLARE OPPERATORS
\DeclareMathOperator{\di}{d\!}

%DECLARE CHARACTERS


\begin{document}
\begin{equation}
	\int \limits_{W_j} c_{j} \di c_{1} \di c_{2} \ldots  \di c_{m} 
	= \dfrac{1}{\prod\limits_{s=1}^{m}h_{s}}
	\left( \  \int\limits_{b(j,l)}^{a(j,l)} c_{j}  \di c_{j}\right) 
	\prod_{\substack{s=1\\ (s\ne j)}}^m \ 
	\int\limits_{a(s,l)}^{b(s,l)} \!  \di c_{s} 
	= c_{j}^{(l)} 
\label{eq_vector1}
\end{equation}


\end{document}