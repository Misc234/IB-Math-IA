\documentclass[11pt]{article}
\usepackage{graphicx}
\usepackage{amsmath}
\usepackage{amsfonts}
\usepackage[margin=2cm]{geometry}
\usepackage{fancyhdr}
\usepackage{enumitem}
\usepackage{natbib}
\pagestyle{fancy}
\usepackage{float}
\usepackage[utf8]{inputenc}
\numberwithin{equation}{subsection}

\title{Approximation of Experimental Physical Values with Non-typical Conditions of Error}
\date{\today}

\DeclareRobustCommand\dash{%
  \unskip\nobreak\thinspace\textemdash\allowbreak\thinspace\ignorespaces}


\begin{document}
%TITLE
\begin{titlepage}
\begin{center}
\pagenumbering{0}
IB Mathematics Higher Level Internal Assessment on Statistics 
\\ 
\rule{\textwidth}{0.25pt}
\linebreak
\Huge{Approximation of Experimental Physical Values with Non-typical Conditions of Error}
\rule{\textwidth}{0.25pt} \\
[15cm]
\large {David Simon Tetruashvili} \\

\end{center}
\end{titlepage}
\newpage
%TITLE END

%TOC
\tableofcontents
\pagenumbering{arabic}
\newpage
%TOC END

%Body
\section{Introduction}


During a large stretch of my schooling, particularly in the natural sciences, I have been taught how to accurately measure and collect data. After said gathering of data, I have been taught to first graph this data on a set of axis, and then to  approximate this given set of data by creating the so called 'Best Fit Line' (BFL). Practically all Physics SL experiments that have been done during the IB course, to approximate our data, my classmates and I have been told to use something called the 'Least Square Method'(LSM). I have l my Mathematics HL class has breefly gone over the LSM and how it analyses the discreet data given to it without giving much detail on either the theory or the math behind it. This led me to wonder to what extent are the approximations that I do in many if not all of my practical papers (for example the IB Physics IA) are
\\ \\
\section{Mathematical Theory}

\section{Algorithm}

\section{Given Examples}

\section{Conclusion and Reflection}
\newpage

\section*{Biblyography}
\end{document}

